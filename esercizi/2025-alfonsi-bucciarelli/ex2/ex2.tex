
\documentclass{article}
\usepackage{graphicx} % Required for inserting images
\usepackage[margin=3cm]{geometry}

\title{Exercices of Decision Making with Constraint Programming\\  Exercise 2}
\author{Andrea Alfonsi, Stefano Bucciarelli}


\begin{document}
\maketitle

\section{N-Queens}
\begin{tabular}{||c | c c | c c||} 
 \hline
 n & \multicolumn{2}{c|}{Alldifferent GC} &  \multicolumn{2}{c||}{Decomposition} \\ 
 \hline\hline
   & Fails & Time & Fails & Time\\ 
 \hline
 28 & 80,085 & 2s 316msec & 417,027 & 6s 405msec \\
 \hline
  29 & 31,731 & 1s 55msec & 212,257 & 3s 433msec \\
 \hline
  30 & 1,626,587 & 36s 916msec & 7,472,978 & 1m 50s \\
 \hline

\end{tabular} \\ \\

\section{Poster Placement}
\begin{tabular}{||c | c c | c c | c c | c c ||} 
 \hline
 Instance & \multicolumn{2}{c|}{Naïve Model} & \multicolumn{2}{c|}{Naïve Model + Implied}& \multicolumn{2}{c|}{Global } &  \multicolumn{2}{c||}{Global + Implied} \\ 
 \hline\hline
    & Fails & Time & Fails & Time & Fails & Time & Fails & Time\\ 
 \hline
 19x19 & 1,089,671 & 13s 74msec & 1 & 251msec & 300,649 & 3s 395msec & 7,237 & 324msec\\
 \hline
 20x20 & 2,221,151 & 28s 917msec & 7393 & 417msec & 2,030 & 244msec & 943 
& 230msec \\
 \hline

\end{tabular} \\  \\

\begin{enumerate}
    \item Comment briefly on how the solver performance changes from one (decomposed or non-global) model to the global model, along with a justification.\\ \\
    The solver performance significantly improves as we move from the naïve model to the global model with implied constraints. This is primarily due to the increased ability of the solver to reason about the problem more effectively. Global constraints enhance problem-solving efficiency by leveraging frequently occurring patterns within the problem's structure. These constraints represent established methods employing targeted propagation techniques to accelerate performance. This specialized propagation prunes inconsistent values from the possible choices for our variables based on specific constraints, thereby minimizing unproductive searches by eliminating values that would inevitably lead to failure. In contrast to decompositions, global constraints utilize incremental computation, reusing stored data from prior propagation steps to increase efficiency, notably by simplifying the computational complexity compared to decomposition methods.
    \item Why is function \texttt{cumulative} used as an implied constraint? Why is it wrong to use it instead of \texttt{diffn}? \\ \\
    The cumulative constraints in the X and Y axes mandate that at each point in the x-axis, the cumulative height of the rectangles that overlap that point does not exceed the total weight limit; and at each point in the y-axis, the cumulative weight of the rectangles that overlap that point does not exceed the total height limit. This is consistently true, given that each rectangle is within the height and weight limitations, and there are no overlaps.\\
    Cumulative will only check the resource usage in it's single dimension (which is useful and can give good pruning sometimes), it does not ensure the no-overlaps property. The diffn global constraint is the canonical one to use for ensuring no-overlap.


\end{enumerate}

\section{The Sequence Puzzle}
\begin{tabular}{||c | c c | c c | c c | c c ||} 
 \hline
 n & \multicolumn{2}{c|}{Base} & \multicolumn{2}{c|}{Base + Implied} &  \multicolumn{2}{c|}{Global} &  \multicolumn{2}{c||}{Global + Implied} \\ 
 \hline\hline
   & Fails & Time & Fails & Time & Fails & Time & Fails & Time\\ 
 \hline
 500 & 652 & 33s 498msec & 495 & 22s 694msec & 989 & 429msec & 493 & 247msec \\
 \hline
 1000 & 1,362 & 2m 51s & 995 & 1m 5s & 1,989 & 1s 329msec & 993 & 589msec \\
 \hline

\end{tabular} \\ \\

\begin{enumerate}
    \item Going from Base → Global, and going from Base + Implied → Global + Implied: what is the main advantage of using a global constraint? Why? \\  \\

    The fundamental model employs the constraint $forall(i \; in \; 0..(n-1))(x[i] = sum(j \; in \; 0..(n-1))(x[j] = i))$. This meta-constraint operates as follows:
    For every instance where the comparison x[j] = i yields true, a supplementary boolean variable is generated. Consequently, this approach introduces $t$ additional boolean variables and $t$ additional constraints in total.
    In contrast, the global model's structure is significantly simpler:
    It directly utilizes only the $n$ variables within the $x$ array.
    It employs a single, encompassing global constraint, specifically designed for efficient cardinality verification.
    Consequently, despite exhibiting a higher number of search failures, the global model achieves superior performance by minimizing the total number of variables and constraints within the problem formulation
    
    \item Is there an implied constraint that now becomes redundant in
the Global + Implied model? Why? \\

    The constraint \texttt{sum(x) = n} becomes unnecessary due to the way the \texttt{global\_cardinality constraint} function operates. The function receives the counts array, which dictates the distribution of values. In this specific scenario, counts is equivalent to $x$, effectively making $x$ the frequency distribution for itself. Consequently, the sum of all frequencies within $x$—from $x[0]$ to $x[n-1]$—is inherently constrained to equal the size of the $x$ array

\end{enumerate}


\end{document}